% karteikarten_controlling.tex
\documentclass{article}
\begin{document}

\section{Was ist Controlling?}
\subsection{Controlling bedeutet Steuerung, Koordination und Informationsversorgung zur Unterstützung der Unternehmensführung; Ziel ist die Sicherstellung rationaler Entscheidungen.}
\hline

\section{Worin unterscheidet sich Controlling vom Rechnungswesen?}
\subsection{Rechnungswesen erfasst und berichtet vergangenheitsorientiert; Controlling wertet aus, plant, steuert und ist zukunftsorientiert (Entscheidungsunterstützung).}
\hline

\section{Welche Hauptrollen hat ein Controller laut ICV-Modell?}
\subsection{Navigator (strategische Richtung aufzeigen), Business Partner (eng mit Management zusammenarbeiten) und Informationslieferant (Reports, Analysen bereitstellen).}
\hline

\section{Was versteht man unter dem Controlling-Zyklus?}
\subsection{Planung $\rightarrow$ Durchführung $\rightarrow$ Kontrolle (Soll-Ist-Vergleich) $\rightarrow$ Informationsbereitstellung $\rightarrow$ Steuerung/Anpassung (zyklischer Prozess).}
\hline

\section{Was sind Sachziele und Formalziele?}
\subsection{Sachziele = inhaltliche Ziele (z. B. Marktanteil, Produktqualität); Formalziele = ökonomische Ziele zur Beurteilung der Effizienz (z. B. Rentabilität, Kostenminimierung).}
\hline

\section{Nenne die Planungsansätze Top-down, Bottom-up und Gegenstromverfahren.}
\subsection{Top-down: Vorgaben vom Management; Bottom-up: Vorschläge aus unteren Ebenen; Gegenstrom: Kombination beider Ansätze (Managementvorgaben + Mitarbeitereinbindung).}
\hline

\section{Welche Kennzahlen sind im operativen Controlling besonders wichtig?}
\subsection{Beispiele: ROI (Return on Investment), Deckungsbeitrag, Break-even-Point, Umsatz, Kostenarten/Kostenstellen/Kostenträger-Kennzahlen.}
\hline

\section{Formel: ROI.}
\subsection{ROI = (Gewinn / eingesetztes Kapital) $\times$ 100\%. (Alternative Schreibweise: Gewinn / Umsatz $\times$ Umsatz / Kapital).}
\hline

\section{Was ist Deckungsbeitrag (DB) und wofür wird er genutzt?}
\subsection{DB = Erlöse - variable Kosten. Er zeigt, wie viel zur Deckung der Fixkosten und zum Gewinn beiträgt; Basis für kurzfristige Entscheidungsfindung (z. B. Annahme eines Zusatzauftrags).}
\hline

\section{Unterschied Vollkostenrechnung vs. Teilkostenrechnung (Direct Costing).}
\subsection{Vollkostenrechnung verteilt alle Kosten auf Produkte; Teilkostenrechnung berücksichtigt nur variable Kosten für Entscheidungsfragen, liefert oft praktischere Aussagen für kurzfristige Entscheidungen.}
\hline

\section{Was ist Break-even-Point?}
\subsection{Punkt, an dem Erlöse = Gesamtkosten (keine Deckungsbeiträge für Gewinn verbleiben). Berechnung: Fixkosten / (Preis - variable Kosten pro Einheit).}
\hline

\section{Wozu dient die Prozesskostenrechnung?}
\subsection{Zuordnung von Gemeinkosten zu Produkten/Dienstleistungen basierend auf tatsächlichen Prozessen; sinnvoll bei komplexen Prozessen oder hohem Gemeinkostenanteil.}
\hline

\section{Was ist Forecasting (rollierende Planung)?}
\subsection{Fortlaufende Aktualisierung von Prognosen (z. B. Quartalsweise), sodass Planung stets einen festgelegten Prüfzeitraum in die Zukunft abdeckt; verbessert Reaktionsfähigkeit.} \\
\hline

\section{Was umfasst strategisches Controlling?}
\subsection{Langfristige Sicherung von Wettbewerbsvorteilen: Umwelt- und Wettbewerbsanalyse, Potenzialentwicklung, Strategiefindung, Instrumente wie SWOT, GAP, Produktlebenszyklus und Portfolio-Analysen.} \\
\hline

\section{Was ist eine SWOT-Analyse?}
\subsection{Analyse von internen Stärken (Strengths) und Schwächen (Weaknesses) sowie externen Chancen (Opportunities) und Risiken (Threats).}
\hline

\section{Was zeigt die BCG-Matrix (Portfolio-Analyse)?}
\subsection{Einordnung von Produkten/ Geschäftsbereichen nach relativer Marktanteil (hoch/niedrig) und Marktwachstum (hoch/niedrig) in: Stars, Question Marks, Cash Cows, Dogs.}
\hline

\section{Was besagt die Kostenerfahrungskurve?}
\subsection{Mit jeder Verdopplung der kumulativen Produktionsmenge sinken die Stückkosten typischerweise um einen Prozentsatz (ca. 20–30\%), abhängig von Branche und Lernrate.}
\hline

\section{Wofür dient die Balanced Scorecard (BSC)?}
\subsection{Verknüpft Strategie mit operativ messbaren Kennzahlen in vier Perspektiven: Finanzen, Kunden, interne Prozesse, Lernen \& Entwicklung; übersetzt Strategie in konkrete Maßnahmen und KPIs.}
\hline

\section{Was ist eine Abweichungsanalyse?}
\subsection{Analyse von Soll-Ist-Abweichungen (z. B. in Plankostenrechnung) zur Identifikation von Ursachen und Ableitung von Korrekturmaßnahmen.}
\hline

\section{Welche typischen Prüfungsfragen solltest du parat haben?}
\subsection{Beispiele: Unterschied Controlling vs. Rechnungswesen; Rollen des Controllers; Berechnung und Interpretation von ROI, Deckungsbeitrag, Break-even; Vor-/Nachteile Vollkosten vs. Teilkosten.}
\hline

\end{document}
