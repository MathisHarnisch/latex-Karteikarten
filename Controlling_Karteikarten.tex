
\documentclass{article}
\begin{document}

\section{Was sind die Hauptaufgaben des Controllings?}
\subsection{Die beiden zentralen Aufgaben des Controllings sind Koordination und Informationsversorgung. Koordination sorgt für die Abstimmung zwischen den Führungsteilsystemen, während Informationsversorgung sicherstellt, dass Entscheidungsträger über relevante, entscheidungsorientierte Daten verfügen.}
\hline

\section{Was versteht man unter der Koordinationsaufgabe im Controlling?}
\subsection{Sie umfasst die sachliche, zeitliche und verhaltensbezogene Abstimmung aller Führungshandlungen. Man unterscheidet zwischen systembildender Koordination (Aufbau und Gestaltung von Planungs-, Kontroll- und Informationssystemen) und systemkoppelnder Koordination (laufende Anpassung bereits bestehender Systeme).}
\hline

\section{Was bedeutet systembildende Koordination?}
\subsection{Das Controlling gestaltet das Führungssystem des Unternehmens aktiv mit. Es entwickelt Strukturen und Instrumente wie Kennzahlensysteme, Budgetierungen oder Verrechnungspreissysteme, um eine effiziente Planung und Kontrolle zu ermöglichen.}
\hline

\section{Was bedeutet systemkoppelnde Koordination?}
\subsection{Das Controlling passt bestehende Systeme laufend an neue Gegebenheiten an. Dies geschieht meist dezentral, beispielsweise durch persönliche Weisungen oder Selbstabstimmung, um auf unvorhersehbare Störungen zu reagieren.}
\hline

\section{Was ist die Informationsaufgabe des Controllings?}
\subsection{Das Controlling stellt sicher, dass relevante Informationen entscheidungsorientiert bereitgestellt werden. Dazu gehört der Aufbau modularer Reportingsysteme, die Nutzung von Entscheidungsunterstützungssystemen (DSS) sowie Top-Management-Informationssystemen (EIS).}
\hline

\section{Was sind Entscheidungsunterstützungssysteme (DSS)?}
\subsection{DSS bestehen aus einer Datenbank, einer Modellbank und einer Methodenbank. Sie helfen, komplexe Entscheidungen zu treffen, z. B. in der Produktionsplanung, Investitionsrechnung oder Tourenoptimierung.}
\hline

\section{Was sind Top-Management-Informationssysteme (EIS)?}
\subsection{EIS dienen der strategischen Unternehmenssteuerung. Ein typisches Beispiel ist die Balanced Scorecard mit den vier Perspektiven: Finanzen, Kunden, Geschäftsprozesse und Lernen und Innovation.}
\hline

\section{Wie ist die Informationssystempyramide nach Chamoni und Gluchowski aufgebaut?}
\subsection{Sie zeigt die Hierarchie betrieblicher Informationssysteme: Basis: Operative Systeme (Kostenrechnung, Buchhaltung), Mitte: Entscheidungsunterstützungssysteme (DSS), Spitze: Strategische Systeme (EIS). Je höher die Ebene, desto stärker ist die Informationsverdichtung.}
\hline

\section{Was ist das Ziel der Unternehmensgesamtplanung?}
\subsection{Die Verknüpfung aller Teilpläne (Absatz, Produktion, Beschaffung, Finanzen, Personal) zu einem integrierten Gesamtplan. Dadurch wird eine einheitliche und widerspruchsfreie Planung ermöglicht.}
\hline

\section{Was unterscheidet eine sequenzielle von einer simultanen Planung?}
\subsection{Bei der sequenziellen Planung werden Teilpläne nacheinander erstellt. Eine Änderung führt oft zu einem erneuten Planungsdurchlauf. Die simultane Planung berücksichtigt alle Parameter gleichzeitig in einer integrierten Modellrechnung.}
\hline

\section{Wie können Planwerte abgeleitet werden?}
\subsection{Typische Verfahren zur Planwertgenerierung sind: Verwendung von Vorjahreswerten, Prozentuale Veränderungen, Prognosen, Kennzahlenbasierte Ableitungen, Nutzung von Verweilzeiten (z. B. 80 Prozent der Umsätze führen zu Einzahlungen).}
\hline

\section{Was versteht man unter der Verweilzeit bei der Planwertgenerierung?}
\subsection{Die Verweilzeit beschreibt den Anteil der Umsatzerlöse, der im selben Zeitraum zu Einzahlungen führt. Beispiel: Bei einer Verweilzeit von $80\%$ werden von $100$ Euro Umsatz $80$ Euro als Einzahlung realisiert.}
\hline

\section{Was sind typische Einsatzgebiete von Management-Informationssystemen (MIS)?}
\subsection{Einsatzgebiete sind: Planung (Budgetierung, Forecasting, Investitionsplanung), Governance und Compliance (Konsolidierung, Risikomanagement), Analyse und Kontrolle (Reporting, Maßnahmencontrolling).}
\hline

\section{Welche fünf Dimensionen des Controllings gibt es?}
\subsection{Nach Küpper werden folgende Dimensionen unterschieden: Organisation, Funktion, Tätigkeit, Steuerungsgrößen und Periodizität.}
\hline

\section{Wie hängen Planung, Kontrolle und Information zusammen?}
\subsection{Planung legt Ziele und Maßnahmen fest, Kontrolle vergleicht Ist- und Soll-Werte, und Information versorgt beide Prozesse mit relevanten Daten. Das Controlling koordiniert diesen Regelkreis zur Steuerung des Unternehmens.}
\hline

\section{Was ist das Ziel der Balanced Scorecard?}
\subsection{Die Balanced Scorecard verbindet finanzielle und nicht-finanzielle Kennzahlen, um die strategische Unternehmensführung zu unterstützen. Sie schafft ein Gleichgewicht zwischen kurz- und langfristigen Zielen.}
\hline

\section{Wie unterstützt das Controlling die Unternehmensführung praktisch?}
\subsection{Durch Bereitstellung entscheidungsrelevanter Informationen, Entwicklung von Kennzahlensystemen, Planung, Budgetierung und kontinuierliche Überwachung der Unternehmensziele.}
\hline

\end{document}
