\documentclass{article}
\begin{document}

\section{Was sind die Hauptaufgaben des Controllings?}
\subsection{Die beiden zentralen Aufgaben des Controllings sind \textbf{Koordination} und \textbf{Informationsversorgung}. 
Koordination sorgt für die Abstimmung zwischen den Führungsteilsystemen, während Informationsversorgung sicherstellt, dass Entscheidungsträger über relevante, entscheidungsorientierte Daten verfügen.}
\hline

\section{Was versteht man unter der \textbf{Koordinationsaufgabe} im Controlling?}
\subsection{Sie umfasst die sachliche, zeitliche und verhaltensbezogene Abstimmung aller Führungshandlungen. 
Man unterscheidet zwischen \textbf{systembildender Koordination} (Aufbau und Gestaltung von Planungs-, Kontroll- und Informationssystemen) und \textbf{systemkoppelnder Koordination} (laufende Anpassung bereits bestehender Systeme).}
\hline

\section{Was bedeutet \textbf{systembildende Koordination}?}
\subsection{Das Controlling gestaltet das Führungssystem des Unternehmens aktiv mit. 
Es entwickelt Strukturen und Instrumente wie \textbf{Kennzahlensysteme}, \textbf{Budgetierungen} oder \textbf{Verrechnungspreissysteme}, um eine effiziente Planung und Kontrolle zu ermöglichen.}
\hline

\section{Was bedeutet \textbf{systemkoppelnde Koordination}?}
\subsection{Das Controlling passt bestehende Systeme laufend an neue Gegebenheiten an. 
Dies geschieht meist dezentral, beispielsweise durch persönliche Weisungen oder Selbstabstimmung, um auf unvorhersehbare Störungen zu reagieren.}
\hline

\section{Was ist die \textbf{Informationsaufgabe} des Controllings?}
\subsection{Das Controlling stellt sicher, dass relevante Informationen entscheidungsorientiert bereitgestellt werden. 
Dazu gehört der Aufbau modularer \textbf{Reportingsysteme}, die Nutzung von \textbf{Entscheidungsunterstützungssystemen (DSS)} sowie \textbf{Top-Management-Informationssystemen (EIS)}.}
\hline

\section{Was sind \textbf{Entscheidungsunterstützungssysteme (DSS)}?}
\subsection{DSS bestehen aus einer \textbf{Datenbank}, einer \textbf{Modellbank} und einer \textbf{Methodenbank}. 
Sie helfen, komplexe Entscheidungen zu treffen, z. B. in der Produktionsplanung, Investitionsrechnung oder Tourenoptimierung.}
\hline

\section{Was sind \textbf{Top-Management-Informationssysteme (EIS)}?}
\subsection{EIS dienen der strategischen Unternehmenssteuerung. 
Ein typisches Beispiel ist die \textbf{Balanced Scorecard} mit den vier Perspektiven: 
\textbf{Finanzen}, \textbf{Kunden}, \textbf{Geschäftsprozesse} und \textbf{Lernen und Innovation}.}
\hline

\section{Wie ist die \textbf{Informationssystempyramide} nach Chamoni/Gluchowski aufgebaut?}
\subsection{Sie zeigt die Hierarchie betrieblicher Informationssysteme:
\begin{itemize}
\item Basis: Operative Systeme (Kostenrechnung, Buchhaltung)
\item Mitte: Entscheidungsunterstützungssysteme (DSS)
\item Spitze: Strategische Systeme (EIS)
\end{itemize}
Je höher die Ebene, desto stärker ist die Informationsverdichtung.}
\hline

\section{Was ist das Ziel der \textbf{Unternehmensgesamtplanung}?}
\subsection{Die Verknüpfung aller Teilpläne (Absatz, Produktion, Beschaffung, Finanzen, Personal) zu einem integrierten Gesamtplan. 
Dadurch wird eine einheitliche und widerspruchsfreie Planung ermöglicht.}
\hline

\section{Was unterscheidet eine \textbf{sequenzielle} von einer \textbf{simultanen} Planung?}
\subsection{Bei der sequenziellen Planung werden Teilpläne nacheinander erstellt. 
Eine Änderung führt oft zu einem erneuten Planungsdurchlauf. 
Die simultane Planung berücksichtigt alle Parameter gleichzeitig in einer integrierten Modellrechnung.}
\hline

\section{Wie können \textbf{Planwerte} abgeleitet werden?}
\subsection{Typische Verfahren zur Planwertgenerierung sind:
\begin{itemize}
\item Verwendung von Vorjahreswerten
\item Prozentuale Veränderungen
\item Prognosen
\item Kennzahlenbasierte Ableitungen
\item Nutzung von Verweilzeiten (z. B. 80\% der Umsätze führen zu Einzahlungen)
\end{itemize}}
\hline

\section{Was versteht man unter der \textbf{Verweilzeit} bei der Planwertgenerierung?}
\subsection{Die Verweilzeit beschreibt den Anteil der Umsatzerlöse, der im selben Zeitraum zu Einzahlungen führt. 
Beispiel: Bei einer Verweilzeit von $80\%$ werden von $100\,€$ Umsatz $80\,€$ als Einzahlung realisiert.}
\hline

\section{Was sind typische Einsatzgebiete von \textbf{Management-Informationssystemen (MIS)}?}
\subsection{Einsatzgebiete sind:
\begin{itemize}
\item \textbf{Planung}: Budgetierung, Forecasting, Investitionsplanung
\item \textbf{Governance / Compliance}: Konsolidierung, Risikomanagement
\item \textbf{Analyse / Kontrolle}: Reporting, Maßnahmencontrolling
\end{itemize}}
\hline

\section{Welche \textbf{fünf Dimensionen des Controllings} gibt es?}
\subsection{Nach Küpper werden folgende Dimensionen unterschieden:
\begin{enumerate}
\item Organisation
\item Funktion
\item Tätigkeit
\item Steuerungsgrößen
\item Periodizität
\end{enumerate}}
\hline

\section{Wie hängen \textbf{Planung}, \textbf{Kontrolle} und \textbf{Information} zusammen?}
\subsection{Planung legt Ziele und Maßnahmen fest, Kontrolle vergleicht Ist- und Soll-Werte, und Information versorgt beide Prozesse mit relevanten Daten. 
Das Controlling koordiniert diesen Regelkreis zur Steuerung des Unternehmens.}
\hline

\section{Was ist das Ziel der \textbf{Balanced Scorecard}?}
\subsection{Die Balanced Scorecard verbindet finanzielle und nicht-finanzielle Kennzahlen, um die strategische Unternehmensführung zu unterstützen. 
Sie schafft ein Gleichgewicht (\emph{Balance}) zwischen kurz- und langfristigen Zielen.}
\hline

\section{Wie unterstützt das Controlling die Unternehmensführung praktisch?}
\subsection{Durch Bereitstellung entscheidungsrelevanter Informationen, Entwicklung von Kennzahlensystemen, Planung, Budgetierung und kontinuierliche Überwachung der Unternehmensziele.}
\hline

\end{document}
